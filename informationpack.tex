% !TEX TS-program = pdflatex
\documentclass[12pt]{article}

% Package the packages
\usepackage[T1]{fontenc}
\usepackage[utf8]{inputenc}
\usepackage{lmodern}
\usepackage[a4paper, margin=1.25in]{geometry}
\usepackage{enumitem}
% \usepackage[nottoc,numbib]{tocbibind}
% \usepackage[round]{natbib}
% \usepackage{pdfpages}
% -

% Configuration
% Change font to Palatino
\renewcommand{\rmdefault}{ppl}
% Change the list item spacing
\setlist{noitemsep}
% Set the bibliography style
% \bibliographystyle{agsm}
% -

% Definitions
\title{Cyber Essentials Information Pack}
\author{Seiber Team}
\date{\today}
% -

% Document
\begin{document}

% Cover page setup
% \maketitle
% \pagebreak
% \tableofcontents
% \pagebreak
% -

\section*{\centering{Introduction}}
The aim of this information pack is to detail what Cyber Essentials is and why organisations need to acquire Cyber Essentials certification in order to work as third-party suppliers to the Welsh Government. It also provides guidance for identifying what level of risk an organisation is operating at based on the types of data that it is handling. This will help you to establish what level of Cyber Essentials certification will be required to bid for Welsh Government contracts.


\section*{\centering{What is Cyber Essentials?}}
Cyber Essentials is a government-backed and industry-supported scheme which helps organisations of all sizes to measure and improve their resilience to cyber attacks. Cyber Essentials certification is accredited by a number of bodies but this information pack is focused on the Information Assurance for Small and Medium Enterprises (IASME) Consortium route.

In order to help organisations protect themselves against some of the most common threats, Cyber Essentials is presented as five technical controls that your organisation should have in place, these are:
\begin{itemize}
  \item \textbf{Boundary Firewalls} \\to prevent unauthorised access to the internal networks of an organisation
  \item \textbf{Secure Configuration} \\to protect systems against common attacks
  \item \textbf{User Access Control} \\to restrict access to those whose need it
  \item \textbf{Malware Protection} \\to protect systems from malware
  \item \textbf{Patch Management} \\to update operating system and third-party software in order to reduce the vulnerability profile of the organisation
\end{itemize}

There are two levels of Cyber Essentials certification: Cyber Essentials and Cyber Essentials PLUS. Organisations are encouraged to recertify once per year, or more frequently in cases where it is necessary to meet specific customer or procurement requirements.

\subsection*{Cyber Essentials}
For the basic level of Cyber Levels, an organisation needs to complete a self-assessment questionnaire. The responses are then independently reviewed by an external certifying body, such as IASME.

The IASME certification for Cyber Essentials costs £300 (+VAT) and has a usual turnaround of 1-3 working days from the time of submission of the self-assessment questionnaire. This cost includes:
\begin{itemize}
  \item Certification against basic level Cyber Essentials
  \item Automatic cyber liability insurance for UK domiciled organisations with a turnover of less than £20m who pass the assessment \textit{(terms apply)}.
\end{itemize}

\subsection*{Cyber Essentials PLUS}
Cyber Essentials PLUS covers the same requirements as the basic Cyber Essentials certification but is assessed by an external certifying body. This provides a higher level of assurance as the testing regime is independent and the certifying body will test the security of an organisation's systems using a range of tools and techniques.


\section*{\centering{Why should I get Cyber Essentials?}}

\section*{\centering{What risk category do I fall into?}}

\section*{\centering{What level of Cyber Essentials do I need?}}


\end{document}
