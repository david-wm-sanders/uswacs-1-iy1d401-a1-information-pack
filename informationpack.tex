% !TEX TS-program = pdflatex
\documentclass[12pt]{article}

% Package the packages
\usepackage[T1]{fontenc}
\usepackage[utf8]{inputenc}
\usepackage{lmodern}
\usepackage[a4paper, margin=0.5in]{geometry}
\usepackage{enumitem}
\usepackage[colorlinks=true, linkcolor=red, citecolor=black, urlcolor=blue]{hyperref}
% \usepackage[nottoc,numbib]{tocbibind}
% \usepackage[round]{natbib}
% \usepackage{pdfpages}
% -

% Configuration
% Change font to Palatino
\renewcommand{\rmdefault}{ppl}
% Change the list item spacing
\setlist{noitemsep}
% Set the bibliography style
% \bibliographystyle{agsm}
% -

% Definitions
\title{Cyber Essentials Information Pack}
\author{Seiber Team}
\date{\today}
% -

% Document
\begin{document}

% Cover page setup
% \maketitle
% \pagebreak
% \tableofcontents
% \pagebreak
% -

% Title
{\centering
  \LARGE Cyber Essentials for SMEs Information Pack\par
  \small Produced by Team Seiber\par
}

\section*{\centering{What is Cyber Essentials?}}
Cyber Essentials is a government-backed and industry-supported scheme which helps organisations of all sizes to measure and improve their resilience to cyber attacks.

Cyber Essentials certification is accredited by a number of bodies but this information pack is focused on the Information Assurance for Small and Medium Enterprises (IASME) Consortium route.

In order to help organisations protect themselves against some of the most common threats, Cyber Essentials is presented as five technical controls that your organisation should have in place, these are:
\begin{itemize}
  \item \textbf{Boundary Firewalls} \\to prevent unauthorised access to the internal networks of an organisation
  \item \textbf{Secure Configuration} \\to protect systems against common attacks
  \item \textbf{User Access Control} \\to restrict access to those whose need it
  \item \textbf{Malware Protection} \\to protect systems from malware
  \item \textbf{Patch Management} \\to update operating system and third-party software in order to reduce the vulnerability profile of the organisation
\end{itemize}

There are two levels of Cyber Essentials certification: Cyber Essentials and Cyber Essentials PLUS. Organisations are encouraged to recertify once per year, or more frequently in cases where it is necessary to meet specific customer or procurement requirements.

\subsection*{Cyber Essentials}
For the basic level of Cyber Levels, an organisation needs to complete a self-assessment questionnaire. The responses are then independently reviewed by an external certifying body, such as IASME.

The IASME certification for Cyber Essentials costs £300 (+VAT) and has a usual turnaround of 1-3 working days from the time of submission of the self-assessment questionnaire. This cost includes:
\begin{itemize}
  \item Certification against basic level Cyber Essentials
  \item Automatic cyber liability insurance for UK domiciled organisations with a\\turnover of less than £20m who pass the assessment \textit{(terms apply)}.
\end{itemize}

\subsection*{Cyber Essentials PLUS}
Cyber Essentials PLUS covers the same requirements as the basic Cyber Essentials certification but is assessed by an external certifying body. This provides a higher level of assurance as the testing regime is independent and the certifying body will test the security of an organisation's systems using a range of tools and techniques.

Organisations have to achieve the basic level of Cyber Essentials before they assessed for Cyber Essentials PLUS. However, with IASME, both assessments can be performed as a part of a singular process.


\section*{\centering{Why should an organisation get Cyber Essentials?}}
\subsection*{The Prevalence of Cyber Attacks and Breaches}
According to \href{https://www.cyberaware.gov.uk/cyberessentials/faq.html}{cyberaware.co.uk}, "45\% of small businesses and 66\% of medium/large businesses reported a cyber breach or attack in the past 12 months. Overall, nearly half of all businesses (46\%) experienced a cyber attack or breach in the past year, causing thousands of pounds worth of costs and disruption to everyday operations."

\subsection*{SMEs}
However, more so than this, Small and Medium-sized Enterprises (SMEs) expect to have a fair and equal opportunity to bid for government contracts and work, alongside larger enterprises, if they wish to do so.

In April 2016, the Welsh Government made it mandatory requirement for third-party suppliers that deal with sensitive information with a \textit{'moderate'} or \textit{'high'} level of risk to be Cyber Essentials certified.

In essence, SMEs that want to be able to bid for new contracts issued by the Welsh Government must be Cyber Essentials certified at a minimum. For contracts that involve sensitive information in large quantities, organisations are required to be Cyber Essentials PLUS certified.


\section*{\centering{What risk category do I fall into?}}
Five levels of risk category have been identified. From the five levels, Cyber Essentials is a mandatory requirement from Level 1 upwards.

\subsection*{The Five Levels of Risk}
\begin{itemize}
  \item \textbf{Level 0} is classed as \textit{'low risk'}. This applies when processing minimal amounts of non-sensitive personal data or when the data is already in the public domain.
  \item \textbf{Level 1} is classed as \textit{'moderate risk'}. This applies when sensitive information may need to be protected. At this level, third-parties working on contracts with low values and small amounts of personal or sensitive data would need to adhere to the basic level of Cyber Essentials.
  \item \textbf{Level 2} applies when dealing with \textit{'sensitive information'}. At this level, a business is required to be Cyber Essentials certified for the duration of the contract.
  \item \textbf{Level 3} requires a business to be Cyber Essentials PLUS certified.
  \item \textbf{Level 4} is classed as \textit{'high risk'}. This level applies to high-value contracts or those that involve dealing with significant amounts of personal or sensitive data. At this level, an organisation is required to obtain ISO27001 as well as Cyber Essentials PLUS.
\end{itemize}


\section*{\centering{What level of Cyber Essentials does an organisation need?}}
All organisations should aim to have at least the basic level of Cyber Essentials. Securing this ahead of time, before bidding for a contract, shows that an organisation values their Cyber Essentials certification beyond it being a means to an end \textit{(i.e. allowing them to bid for a specific contract)}.

However, any organisation that thinks that it might regularly want to bid for contracts that could involve working with sensitive data should seriously consider getting Cyber Essentials PLUS certification. This should be done as soon as possible, as due to the requirement for independent testing the timescale for acquiring certification is longer.

\end{document}
